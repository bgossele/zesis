\chapter{Future work}

Deze thesis biedt nog vele perspectieven voor verder onderzoek. Eerst en vooral is er nog werk om van onze implementatie van GEMINI met Cassandra een afgewerkt product te maken: enkele features uit de oorspronkelijke GEMINI met SQLite kunnen nog toegevoegd worden. De eerste prioriteit is hierbij het gebruiken van custom annotations. Daarnaast kan het querymechanisme van onze implementatie nog verfijnd worden: SQL-\texttt{JOIN}s zijn nog niet ge\"implementeerd en het opstellen en parsen van queries gebeurt nu op een vrij eenvoudige manier, terwijl door het strategischer kiezen van de volgorde van subqueries wellicht nog veel aan performantie gewonnen kan worden. Het ultieme doel hierbij is een dynamische query-planner zoals vele RDBMs en sommige NoSQL-systemen die ook kennen. Bovendien moet het mogelijk zijn de GEMINI queries intelligenter te parsen en op te splitsen in subqueries zodat de originele syntax van GEMINI volledig hersteld kan worden.\\
Een andere feature die ons ontwerp nog gebruiksvriendelijker zou maken, is een tool om automatisch nieuwe hulptabellen te defini\"eren en op te stellen. Zo kan de gebruiker zelf kiezen welke kolommen uit de basistabellen van GEMINI doorzocht kunnen worden.\\
Een laatste verbetering zou het realiseren van een incrementeel uitbreidbare versie van GEMINI zijn: vanuit Janssen Pharmaceutica is de vraag groot naar een versie van GEMINI waarin nieuwe genoomdata ingeladen kan worden zonder de reeds aanwezige gegevens opnieuw mee in te moeten laden. Cassandra laat toe het dataschema te wijzigen en ons ontwerp staat hier dan ook open voor. Dit zou een waarlijke \textit{killer feature} zijn die de tekortkomingen van onze implementatie op het gebied van inladen van genoomdata grotendeels zou kunnen compenseren.\\

Dan rest er nog de vraag of andere systemen niet geschikter zijn voor deze toepassing dan Cassandra: tijdens ons onderzoek hebben we ondervonden dat het datamodel van Cassandra restrictief is op het gebied van queries. We hebben dit in onze implementatie grotendeels kunnen verhelpen door zelf een extra query-engine bovenop Cassandra te bouwen, maar andere systemen bieden deze functionaliteit al van begin af aan. Zo zou het interessant zijn dezelfde denkoefening nog eens te maken met het eerder besproken Cloudera Impala, dat veel uitgebreidere query-functionaliteit biedt. Ook het verder bewandelen van de PostgreSQL-piste, zoals de ontwikkelaars van GEMINI reeds kort gedaan hebben, lijkt interessant. Systemen als CitusDB \cite{citus_db} bouwen bijvoorbeeld een parallelle database bovenop PostgreSQL en bieden veel perspectieven voor schaalbare query-processing.
\\Een laatste, specifiek op genoomanalyse gerichte, Big Data-systeem is de Google Genomics-API\cite{google_genomics} die genoomanalyse-tools gebaseerd op Google's data-analysetools aanbiedt als cloud-service. Een nadeel hieraan is dat farmaceutische bedrijven uit concurrentie- en security-overwegingen niet geneigd zullen zijn hun gegevens uit handen te geven, maar anderzijds biedt dit natuurlijk wel de van Google gekende schaalbaarheid en performantie op maat gemaakt voor de bio-informatica.