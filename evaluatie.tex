\chapter{Evaluatie}

\section{Functionele vereisten: testing}

Onze implementatie van GEMINI in combinatie met Cassandra ondersteunt de belangrijkste features van GEMINI (zie \ref{gemini_beschrijving}), op een paar uitzonderingen na:
\begin{itemize}
\item Inladen genoominformatie: GEMINI met Cassandra ondersteunt dezelfde input (i.e. VCF-, PED-, en annotatiefiles) als GEMINI met SQLite. Enkel door de gebruiker zelf gedefinieerde annotatiebestanden zijn nog niet ondersteund, maar dit is perfect analoog met de voorgedefinieerde bestanden en werd slechts uit tijdgebrek niet ge\"implementeerd.
\item Querying van genoominformatie: GEMINI met Cassandra behoudt de query-functionaliteit van de SQLite-implementatie, inclusief de uitgebreide SQL-syntax zoals genotype-filters en -wildcards, sample-wildcards. De enige soort queries die onze implementatie niet ondersteunt, zijn pure range queries. Dit kan ook opgelost worden met een workaround in de client-code (zie \ref{arbitraire_queries}), die wij echter niet ge\"implementeerd hebben.
\item Voorts biedt GEMINI nog vele andere tools voor zeer specifieke genetische onderzoeksdoeleinden. We hebben die niet nader bestudeerd, maar gezien ze allen voortbouwen op de queryfunctionaliteit, hebben we de basis er wel voor gelegd.
\end{itemize}

Dankzij de uitgebreide verzameling unit-tests van GEMINI hebben we de laad- en queryfunctionaliteiten ook grondig kunnen testen en het correct functioneren van onze implementatie kunnen bewijzen. We hebben hiervoor 82 unit-tests uit de standaard meegeleverde unit-tests van GEMINI gebruikt, die onze implementatie allemaal doorstaat. Daarnaast hebben we de standaard unit-tests voor queries met genotype-wildcards nog herhaald met meerdere cores, om specifiek het parallellisatie-mechanisme van onze implementatie te testen. Ook deze tests waren allemaal succesvol.

\section{Niet-functionele vereisten: benchmarking}

Gezien onze implementatie de gewenste functionaliteit biedt, rest nog de vraag of ze ook, zoals beoogd, beter schaalt naar grote genoomdatasets. Om dit te evalueren hebben we onze versie van GEMINI onderworpen aan een reeks benchmarkingtests en vervolgens de executietijd van zowel het inladen als doorzoeken van genoominformatie met GEMINI gemeten. Ter vergelijking hebben we ook de prestaties van de SQLite-versie van GEMINI gemeten voor dezelfde tests.

\subsection*{Testomgeving}

Voor de experimenten gebruikten we, in samenspraak met de domeinexperten van Janssen Pharmaceutica, de publiek beschikbare genoomdata van het 1000 Genome project \cite{10002012integrated}. Die datasets bevatten enkel de VCF-files, niet de sample-informatie van de proefpersonen. Om toch de hele featureset te kunnen testen, hebben we willekeurig een geslacht en fenotype aan de samples toegekend. Specifiek hebben we 5 verschillende VCF-files (in reeds gezipte toestand, vandaar de bestandsextensie) gebruikt:
\begin{itemize}
\item \texttt{s\_1092.vcf.gz} van 1.8 GB met 494328 variants van 1092 samples.
\item \texttt{m\_1092.vcf.gz} van 3.0 GB met 855166 variants van 1092 samples.
\item \texttt{l\_1092.vcf.gz} van 6.7 GB met 1882663 variants van 1092 samples.
\item \texttt{xl\_1092.vcf.gz} van 11.8 GB met 3307592 variants van 1092 samples.
\item \texttt{s\_2504.vcf.gz} van 1.2 GB met 7081600 variants van 2504 samples.
\end{itemize}

De experimenten hebben we uitgevoerd op de cluster van het lab, bestaande uit nodes met elk 2 Intel X5660-processoren (6 cores, 12 threads, dus 12 cores en 24 threads per node), 96 GB RAM en 500 GB schijfruimte. GEMINI met SQLite draaide steeds in zijn geheel op 1 zo'n node, GEMINI met Cassandra hebben we afhankelijk van experiment tot experiment op 1 of 2 nodes uitgevoerd, met daarnaast een Cassandra-cluster die draaide op 3 tot 10 nodes, weer afhankelijk van het experiment.

\subsection*{Structuur}

In de volgende paragraaf zullen we \'e\'en voor \'e\'en de experimenten en hun resultaten bespreken, te beginnen met de performantietesten van het inladen van gegevens, en vervolgens de queries. We zullen daarbij deze structuur hanteren:
\begin{enumerate}
\item Opzet van het experiment. Wat meet het experiment precies?
\item Omstandigheden: precieze input, commando, gebruikte infrastructuur.
\item Resultaten
\item Reflectie
\end{enumerate}

\subsection{Experiment 1: \texttt{gemini load} met Cassandra vs. SQLite}
\label{exp1}

\subsubsection{Opzet}

Het doel van dit experiment is te meten hoe snel GEMINI genoomdata kan inladen in een Cassandra-databank, en dit te vergelijken met de originele SQLite-versie van GEMINI. Van de SQLite-versie hebben we de totale tijd gemeten, evenals de tijd die GEMINI nodig heeft om de rijen in te laden in de aparte chunk-databases en de tijdsduur van het merge-proces hierna. Van onze implementatie hebben we de ook de totale tijdsduur gemeten, en bovendien elke thread per batch van 50 variants laten loggen hoe lang het invoeren van zo'n batch duurde, om een beter zicht te krijgen op de throughput. Bovendien hebben we voor onze implementatie gemeten hoe lang specifiek het invoeren van de variants in de \texttt{variants}-tabel duurt, maar ook hoe lang het invoeren van de genotype-data in de genotype-hulptabellen (dit zijn er 4) duurt. 

\subsubsection{Omstandigheden}
We hebben de genoomdata uit de \texttt{s\_1092.vcf.gz} en \texttt{m\_1092.vcf.gz} ingeladen in een Cassandra-cluster met 10 nodes en replicatiefactor 3. GEMINI met Cassandra maakte gebruik van 24 threads, verdeeld over 2 nodes. GEMINI met SQLite draaide op 24 threads, op \'e\'en node.

\subsubsection{Resultaten}

\begin{table}[h]
\begin{tabular}{@{}llll@{}}
\cmidrule(l){2-4}
                              & \multicolumn{3}{|l|}{\textbf{SQLite}}                             \\ 
\cmidrule(l){1-4}
\multicolumn{1}{|l|}{\textbf{Dataset}}  & \multicolumn{1}{l|}{\textbf{Chunk-dbs laden [s]}} & \multicolumn{1}{l|}{\textbf{Chunk-dbs mergen [s]}} & \multicolumn{1}{l|}{\textbf{Totaal [s]}} \\ \midrule
\multicolumn{1}{|l|}{\textbf{\texttt{s\_1092.vcf.gz}}} &    \multicolumn{1}{l|}{$2440.47 \pm 7.46$}         & \multicolumn{1}{l|}{$551.95 \pm 18.44$}          & \multicolumn{1}{l|}{$2992.41 \pm 20.92$}                          \\
\multicolumn{1}{|l|}{\textbf{\texttt{m\_1092.vcf.gz}}} &    \multicolumn{1}{l|}{$2945.62 \pm 44.90$}       & \multicolumn{1}{l|}{$880.89 \pm 34.47$}          & \multicolumn{1}{l|}{$3826.51 \pm 76.88$}                          \\
\multicolumn{1}{|l|}{\textbf{\texttt{l\_1092.vcf.gz}}} &    \multicolumn{1}{l|}{$6879.86 \pm 40.39$}       & \multicolumn{1}{l|}{$2021.01 \pm 35.17$}          & \multicolumn{1}{l|}{$8900.86 \pm 63.32$}                          \\
\multicolumn{1}{|l|}{\textbf{\texttt{xl\_1092.vcf.gz}}} &    \multicolumn{1}{l|}{$11437.81 \pm 135.09$}       & \multicolumn{1}{l|}{$3462.00 \pm 70.37$}          & \multicolumn{1}{l|}{$14899.91 \pm 123.45$}                          \\
\bottomrule
\end{tabular}
\caption{Meetresultaten voor de runtime van het inladen van genoomdata in SQLite bij variabele grootte van de genoomdataset.}
\end{table}

\begin{table}[h]

\begin{tabular}{@{}llll@{}}
\cmidrule(l){2-4}
                  & \multicolumn{3}{|l|}{\textbf{Cassandra}}                                    \\ 
\cmidrule(l){1-4}
\multicolumn{1}{|l|}{\textbf{Dataset}}  & \multicolumn{1}{l|}{\textbf{\texttt{INSERT variants} [s]}} & \multicolumn{1}{l|}{\textbf{\texttt{INSERT genotypes} [s]}} & \multicolumn{1}{l|}{\textbf{Tot. duur batch [s]}}             \\ \midrule
\multicolumn{1}{|l|}{\textbf{\texttt{s\_1092.vcf.gz}}} & \multicolumn{1}{l|}{$2.06 \pm 1.19$} & \multicolumn{1}{l|}{$45.23 \pm 4.10$}   & \multicolumn{1}{l|}{$50.91 \pm 4.36$}\\
\multicolumn{1}{|l|}{\textbf{\texttt{m\_1092.vcf.gz}}} & \multicolumn{1}{l|}{$2.22 \pm 0.61$} & \multicolumn{1}{l|}{$45.70 \pm 3.80$}     &  \multicolumn{1}{l|}{$51.55 \pm 3.83$}  \\
\multicolumn{1}{|l|}{\textbf{\texttt{l\_1092.vcf.gz}}} & \multicolumn{1}{l|}{$2.44 \pm 1.26$} & \multicolumn{1}{l|}{$46.95 \pm 4.46$}     &  \multicolumn{1}{l|}{$52.92 \pm 4.55$}  \\
\bottomrule
\end{tabular}
\caption{Meetresultaten voor de duur van het inladen van genoomdata in Cassandra bij variabele grootte van de genoomdataset. De gemeten tijden zijn de waarden voor \'e\'en batch van 50 variants, gemiddeld over 24 threads.}
\end{table}

\subsubsection{Reflectie}

Het inladen van de genoomdata verloopt duidelijk ettelijke malen sneller in de originele versie van GEMINI dan de Cassandra-versie. Dit is voornamelijk te wijten aan twee factoren: enerzijds moet GEMINI met Cassandra gigantische veel rijen invoeren dankzij het nieuwe datamodel. De hulptabellen met gegevens over de genotypes van elke sample per variant hebben $\#variants * \#samples$ rijen, wat zelfs voor de kleinste dataset al oploopt tot om en bij de 500 miljoen rijen. Uit de metingen blijkt ook duidelijk dat het vullen van deze tabellen het meeste tijd in beslag neemt: zo'n 90\% van het totale laadproces. Met de bedenking erbij dat voor elke 50 variants, $4 * 50 * 1092$ rijen worden ingevoerd, en dit door 24 threads tegelijkertijd, haalt onze implementatie in deze setup voor deze hulptabellen een schrijfthroughput van in de orde van 100000 writes/s. Bovendien varieert de throughput nauwelijks met toenemende grootte (in variants) van de ingevoerde dataset. Om die reden, en omwille van de extreem lange duur, hebben we het experiment ook niet herhaald voor de \texttt{xl\_1092.vcf.gz}-dataset. Een schatting leert dat het inladen hiervan in deze setup ongeveer 40 uur zou duren.\\
Een belangrijke bemerking bij de snelheid van het inladen is dat het datamodel van onze implementatie, in tegenstelling tot de SQLite-implementatie, uitbreidbaar is: het is niet nodig elke keer extra genoomdata beschikbaar is, de volledige dataset opnieuw in te laden. Hoewel onze implementatie trager is, zou dit een eenmalige kost moeten zijn.\\

Het inladen in Cassandra verliep tijdens de experimenten niet zonder horten of stoten: af en toe traden timeouts op in de \texttt{INSERT}-queries, die altijd opgevangen konden worden door dezelfde query opnieuw te proberen (desnoods meerdere keren). Er zijn twee uitzonderingen die niet zomaar hersteld konden worden. Ten eerste het invoeren van zeer lange (strings van 10000'en of 100000'en karakters) genotypes in de \texttt{variants\_by\_samples\_gts}-tabel en ten tweede het invoeren van variants uit de \texttt{s\_2504.vcf.gz}-dataset in de \texttt{variants}-tabel.\\
Van de eerste soort komen er exact 5549 voor in de \texttt{s\_1092.vcf.gz}-dataset (op ongeveer 500 miljoen genotypes van samples) en 3233 in de \texttt{m\_1092.vcf.gz}-dataset (op meer dan 900 miljoen genotypes), die steevast voor een error bij het inladen zorgen.\\
Fouten van de tweede soort, in de dataset met 2504 samples, traden op omwille van de zeer lange \texttt{INSERT}-queries, met kolommen voor de verschillende genotype-eigenschappen van elke sample. Dit impliceert een sterke beperking op het aantal samples in de dataset. Wij hebben dit niet verder onderzocht, maar geloven dat hier zeker oplossingen voor te vinden zijn in toekomstig onderzoek.

  
\subsection{Experiment 2: Schaalbaarheid \texttt{gemini load} met grootte Cassandra-cluster}
\label{exp2}

\subsubsection{Opzet}

Dit experiment bestudeert de invloed van de grootte van het Cassandra-cluster op de duur van het inladen van genoomdata. We hebben wederom de duur van het inladen van elke batch van 50 variants gemeten, met daarin nog een opsplitsing tussen de tijd nodig voor het invoeren van de variants in de \texttt{variants}-tabel en het invoeren van samplegenotypes in 4 hulptabellen van de \texttt{variants}-tabel.

\subsubsection{Omstandigheden}

We hebben de performantie gemeten voor de \texttt{s\_1092.vcf.gz}-dataset, in Cassandra-clusters van 5, 7 en 10 nodes. GEMINI draaide op 24 threads verspreid over 2 nodes.

\subsubsection{Resultaten}

\begin{table}[h]

\begin{tabular}{@{}llll@{}}
\cmidrule(l){1-4}
\multicolumn{1}{|l|}{\textbf{\# Nodes}}  & \multicolumn{1}{l|}{\textbf{\texttt{INSERT variants} [s]}} & \multicolumn{1}{l|}{\textbf{\texttt{INSERT genotypes} [s]}} & \multicolumn{1}{l|}{\textbf{Tot. duur batch [s]}}             \\ \midrule
\multicolumn{1}{|l|}{5} & \multicolumn{1}{l|}{$2.44 \pm 1.38$} & \multicolumn{1}{l|}{$54.54 \pm 4.51$}   & \multicolumn{1}{l|}{$60.54 \pm 4.70$}\\
\multicolumn{1}{|l|}{7} & \multicolumn{1}{l|}{$1.98 \pm 1.12$} & \multicolumn{1}{l|}{$49.30 \pm 3.77$}     &  \multicolumn{1}{l|}{$54.89 \pm 4.17$}  \\
\multicolumn{1}{|l|}{10} & \multicolumn{1}{l|}{$2.06 \pm 1.19$} & \multicolumn{1}{l|}{$45.23 \pm 4.10$}     &  \multicolumn{1}{l|}{$50.91 \pm 4.36$}  \\
\multicolumn{1}{|l|}{12} & \multicolumn{1}{l|}{$2.29 \pm 1.29$} &\multicolumn{1}{l|}{$44.26 \pm 4.32$}     &  \multicolumn{1}{l|}{$50.16 \pm 4.46$}  \\
\bottomrule
\end{tabular}
\caption{Meetresultaten voor de duur van het inladen van genoomdata in Cassandra bij variabele grootte van het Cassandra-cluster. De gemeten tijden zijn de waarden voor \'e\'en batch van 50 variants, gemiddeld over 24 threads.}
\end{table}

\subsubsection{Reflectie}

De totale throughput stijgt merkbaar bij toenemende grootte van de cluster. Dit laat zich vooral merken bij het invoegen van de genotypes van de samples: gezien dit per batch veel meer writes inhoudt  (1092x meer) dan het invoeren van de variants in de  \texttt{variants}-tabel, laat het effect zich hier meer voelen. Bij stijging van 10 naar 12 nodes treedt er verzadiging op en stijgt de throughput niet meer zo drastisch als bij de overstap van 5 naar 7 of van 7 naar 10 nodes. Bij 10 nodes is de totale schrijfbandbreedte al voldoende groot en ligt de bottleneck weer bij GEMINI, eerder dan bij Cassandra.

\subsection{Experiment 3: \texttt{gemini query} met Cassandra vs. SQLite}
\label{exp3}

\subsubsection{Opzet}
Het doel van dit experiment is de executietijd van verschillende relevante queries in GEMINI te vergelijken van de Cassandra- en SQLite-implementaties. Daarvoor hebben we 7 queries geselecteerd die de volledige queryingfunctionaliteit van GEMINI bestrijken en bovendien representatief zijn voor de queries die biologen in GEMINI uitvoeren. De gebruikte queries zijn gebaseerd op de documentatie van GEMINI en hun relevantie is afgetoetst bij onze contactpersoon bij Janssen Pharmaceutica.

\subsubsection{Omstandigheden}

We hebben de runtime van onderstaande 8 queries gemeten op de \texttt{s\_1092.vcf}-dataset op een Cassandra-cluster van 9 nodes met replicatiefactor 3 en van queries 1, 4 en 8 op de \texttt{m\_1092.vcf}-dataset op een Cassandra-cluster van 10 nodes met replicatiefactor 3. Ter vergelijking voerden we dezelfde queries ook uit met de SQLite-versie van GEMINI. Onze implementatie van GEMINI hebben we alle 24 beschikbare threads op \'e\'en node laten gebruiken. De SQLite-versie van GEMINI biedt geen opties om queries parallel of op een andere manier sneller te laten verlopen, dit is dus snelste configuratie. Voor de Cassandra-versie van GEMINI hebben we zowel gemeten hoe lang het uitvoeren van de subqueries als het totale proces duurt, voor de SQLite-versie enkel de totale duur.

\begin{enumerate}
\item \textbf{Query 1} Deze query dient om de performantie van eenvoudige queries op de kolommen van een tabel zoals de \texttt{variants}-tabel te meten.

\begin{adjustwidth}{-1.0in}{}
\lstinputlisting[language=sh,firstline=1,lastline=3]{gemini_src/benchmarking_queries.sh}
\end{adjustwidth}

\item \textbf{Query 2} Deze query dient ook om de performantie van eenvoudige queries op de kolommen van de \texttt{variants}-tabel te meten.

\begin{adjustwidth}{-1.0in}{}
\lstinputlisting[language=sh,firstline=25,lastline=28]{gemini_src/benchmarking_queries.sh}
\end{adjustwidth}

\item \textbf{Query 3} Deze query dient om de performantie van een eenvoudige genotype-filter te meten, met de disjuncte combinatie van voorwaarden op het genotype van 2 proefpersonen.

\begin{adjustwidth}{-1.0in}{}
\lstinputlisting[language=sh,firstline=5,lastline=6]{gemini_src/benchmarking_queries.sh}
\end{adjustwidth}

\item \textbf{Query 4} Deze query dient om de performantie van een eenvoudige genotype-filter te meten, met de conjuncte combinatie van voorwaarden op het genotype van 2 proefpersonen.

\begin{adjustwidth}{-1.0in}{}
\lstinputlisting[language=sh,firstline=8,lastline=9]{gemini_src/benchmarking_queries.sh}
\end{adjustwidth}

\item \textbf{Query 5} Deze query dient om de performantie van een eenvoudige genotype-filter te meten, met de negatie van een voorwaarde op het genotype van 1 sample.

\begin{adjustwidth}{-1.0in}{}
\lstinputlisting[language=sh,firstline=11,lastline=12]{gemini_src/benchmarking_queries.sh}
\end{adjustwidth}

\item \textbf{Query 6} Deze query dient om de performantie van een eenvoudige genotype-wildcard te meten, die alle variants zoekt waarvoor alle geaffecteerde (fenotype = 2) samples niet homozygoot voor het referentie-allel zijn.

\begin{adjustwidth}{-1.0in}{}
\lstinputlisting[language=sh,firstline=14,lastline=15]{gemini_src/benchmarking_queries.sh}
\end{adjustwidth}

\item \textbf{Query 7} Deze query dient om de performantie van een combinatie van genotype-wildcards te meten, die alle variants zoekt waarvoor alle geaffecteerde (fenotype = 2) samples wel homozygoot voor het referentie-allel zijn en bovendien de depth van de variant call voldoende hoog is.

\begin{adjustwidth}{-1.0in}{}
\lstinputlisting[language=sh,firstline=17,lastline=19]{gemini_src/benchmarking_queries.sh}
\end{adjustwidth}

\item \textbf{Query 8} Deze query dient om de performantie van een combinatie van genotype-wildcards te meten, die alle variants zoekt waarvoor alle geaffecteerde (fenotype = 2) samples niet homozygoot voor het referentie-allel zijn en bovendien de depth van de variant call voor maximum 9 samples lager dan 20 is. Het is een minder restricitieve versie van query 7.

\begin{adjustwidth}{-1.0in}{}
\lstinputlisting[language=sh,firstline=21,lastline=23]{gemini_src/benchmarking_queries.sh}
\end{adjustwidth}

\end{enumerate}

\newpage
\subsubsection{Resultaten}

\begin{table}[h]
\begin{tabular}{@{}lllll@{}}
\cmidrule(l){3-5}
                              &         & \multicolumn{2}{|l|}{\textbf{Cassandra}}                                                      & \multicolumn{1}{l|}{\textbf{SQLite}}                             \\ 
\cmidrule(l){2-5}
\multicolumn{1}{l|}{}  & \multicolumn{1}{l|}{\textbf{\# Resultaten}}               & \multicolumn{1}{l|}{\textbf{Subqueries [s]}} & \multicolumn{1}{l|}{\textbf{Totaal [s]}} & \multicolumn{1}{l|}{\textbf{Totaal [s]}} \\ \midrule
\multicolumn{1}{|l|}{\textbf{Query 1}} & \multicolumn{1}{l|}{339819} & \multicolumn{1}{l|}{$6.34 \pm 0.13$}             & \multicolumn{1}{l|}{$443.46 \pm 6.85$}          & \multicolumn{1}{l|}{$148.4 \pm 1.06$}                          \\
\multicolumn{1}{|l|}{\textbf{Query 2}} & \multicolumn{1}{l|}{15588} & \multicolumn{1}{l|}{$1.62 \pm 0.38$}             & \multicolumn{1}{l|}{$40.37 \pm 2.40$}          & \multicolumn{1}{l|}{$2.53 \pm 0.02$}                          \\
\multicolumn{1}{|l|}{\textbf{Query 3}} & \multicolumn{1}{l|}{447392} &\multicolumn{1}{l|}{$8.75 \pm 0.07$}            & \multicolumn{1}{l|}{$573.85 \pm 3.78$}          & \multicolumn{1}{l|}{$8428.81 \pm 30.31$}                        \\
\multicolumn{1}{|l|}{\textbf{Query 4}} & \multicolumn{1}{l|}{15218} &\multicolumn{1}{l|}{$8.74 \pm 0.10$}            & \multicolumn{1}{l|}{$43.19 \pm 1.20$}            & \multicolumn{1}{l|}{$8412.17 \pm 36.02$}                        \\
\multicolumn{1}{|l|}{\textbf{Query 5}} & \multicolumn{1}{l|}{463148} &\multicolumn{1}{l|}{$1.90 \pm 0.06$}            & \multicolumn{1}{l|}{$593.12 \pm 5.09$}           & \multicolumn{1}{l|}{$8405.53 \pm 29.26$}                        \\
\multicolumn{1}{|l|}{\textbf{Query 6}} & \multicolumn{1}{l|}{1802} &\multicolumn{1}{l|}{$219.63 \pm 4.24$}                 & \multicolumn{1}{l|}{$237.22 \pm 4.55$}                    & \multicolumn{1}{l|}{$8404.66 \pm 24.28$}                        \\
\multicolumn{1}{|l|}{\textbf{Query 7}} & \multicolumn{1}{l|}{0} &\multicolumn{1}{l|}{$109.67 \pm 1.12$}          & \multicolumn{1}{l|}{$109.67 \pm 1.12$}           & \multicolumn{1}{l|}{$8513.17 \pm 132.52$}                        \\
\multicolumn{1}{|l|}{\textbf{Query 8}} & \multicolumn{1}{l|}{5} &\multicolumn{1}{l|}{$353.30 \pm 1.68$} & \multicolumn{1}{l|}{$367.29 \pm 2.08$}       & \multicolumn{1}{l|}{$8448.35 \pm 19.25$}                                \\ 
\bottomrule
\end{tabular}
\caption{Meetresultaten voor de executietijd van queries op de \texttt{s\_1092.vcf.gz}-dataset in Cassandra en SQLite.}
\end{table}

\begin{table}[h]
\begin{tabular}{@{}lllll@{}}
\cmidrule(l){3-5}
                              &         & \multicolumn{2}{|l|}{\textbf{Cassandra}}                                                      & \multicolumn{1}{l|}{\textbf{SQLite}}                             \\ 
\cmidrule(l){2-5}
\multicolumn{1}{l|}{}  & \multicolumn{1}{l|}{\textbf{\# Resultaten}}               & \multicolumn{1}{l|}{\textbf{Subqueries [s]}} & \multicolumn{1}{l|}{\textbf{Totaal [s]}} & \multicolumn{1}{l|}{\textbf{Totaal [s]}} \\ \midrule
\multicolumn{1}{|l|}{\textbf{Query 1}} & \multicolumn{1}{l|}{579823} & \multicolumn{1}{l|}{$22.12 \pm 0.54$}             & \multicolumn{1}{l|}{$742.11 \pm 6.25$}          & \multicolumn{1}{l|}{$252.40 \pm 2.19$}                         \\
\multicolumn{1}{|l|}{\textbf{Query 4}} & \multicolumn{1}{l|}{26964} &\multicolumn{1}{l|}{$26.32 \pm 0.22$}            & \multicolumn{1}{l|}{$88.60 \pm 1.71$}            & \multicolumn{1}{l|}{$14553.05 \pm 73.92$}                         \\
\multicolumn{1}{|l|}{\textbf{Query 8}} & \multicolumn{1}{l|}{3} &\multicolumn{1}{l|}{$1276.65 \pm 13.88$} & \multicolumn{1}{l|}{$1305.21 \pm 11.29$}       & \multicolumn{1}{l|}{$14611.31 \pm 71.27$}                                \\ 
\bottomrule
\end{tabular}
\caption{Meetresultaten voor de executietijd van queries op de \texttt{m\_1092.vcf.gz}-dataset in Cassandra en SQLite.}
\end{table}

\subsubsection{Reflectie}

Uit het resultaat voor query 1 en 2 blijkt dat GEMINI met Cassandra voor eenvoudige queries op de \texttt{variants}-tabel het grootste deel van de totale executietijd besteed aan het ophalen van het finale resultaat. GEMINI met SQLite moet hiervoor niet over het netwerk communiceren en is duidelijk sneller.\\
Queries 3, 4 en 5 tonen dat de evaluatie van genotype-filters in GEMINI met Cassandra grootteordes sneller verloopt dan met SQLite: ongeacht de grootte van het resultaat zal SQLite \'e\'en voor \'e\'en alle variants overlopen, de binaire genotype-kolommen decomprimeren en de filter als een Python-\texttt{eval}-statement evalueren. GEMINI met Cassandra komt ervan af met een beperkt aantal (gelijk aan het aantal clausules in de filter) subqueries die elk slechts een subset van het totale variants teruggeven, en moet enkel de resultaten hiervan nog combineren. De totale kost hangt hier duidelijk samen met de kardinaliteit van het resultaat, waar die in het SQLite afhangt van het totale aantal variants.\\
Queries 6, 7 en 8 tonen dat ook de evaluatie van genotype wildcards significant sneller verloopt met een Cassandra-database dan met een SQLite-database. Het evalueren van de subqueries duurt bij deze queries duidelijk langer dan in de eenvoudige genotype-filters van queries 4 \& 5, maar het zijn er dan ook beduidend meer. Er zijn 566 samples met fenotype 2, wat betekent dat elke wildcard zich vertaalt naar 566 subqueries. \\
Het verschil in uitvoeringstijd tussen query 7 en 8 kan verklaard worden door de kardinaliteit van het resultaat van het linkerlid van de respectievelijke conjuncties: het linkerlid van query 7 geeft 0 variants terug, waardoor de evaluatie van het rechterlid overgeslagen kan worden. Het linkerlid van query 8 levert daarentegen een niet-ledige verzameling van variants op, waardoor het rechterlid wel ge\"evalueerd dient te worden. Dit kost opnieuw 566 subqueries.\\

Voor de grotere \texttt{m\_1092.vcf.gz}-dataset tekent zich een gelijkaardig scenario af: bij query 1 is onze implementatie trager dan GEMINI met SQLite, en besteedt onze versie veruit de meeste tijd aan het ophalen van de finale resultaten. Bij query 4 is er in dit geval zelfs een speedup van 150x waar te nemen tegenover SQLite, bij query 8 ondanks de complexe wildcard-evaluatie nog steeds een speedup van meer dan 10X. 

\subsection{Experiment 4: Schaalbaarheid \texttt{gemini query} bij parallellisatie GEMINI}
\label{exp4}

\subsubsection{Opzet}

Het doel van dit experiment is te onderzoeken of de querysnelheid meeschaalt wanneer GEMINI geparallelliseerd wordt over meerdere threads. We bestuderen het effect van parallellisatie bij zowel het ophalen van het finale resultaat van queries als bij het evalueren van subqueries van genotype-wildcards.

\subsubsection{Omstandigheden}

Voor dit experiment hebben we query 1 en 8 uit experiment 3 (\ref{exp3}) hergebruikt: query 1 omwille van de grootte van het resultaat, query 8 omwille van de duur van de evaluatie van de genotype wildcard. We hebben de uitvoeringstijd voor de twee queries gemeten bij uitvoering van GEMINI met 4, 8, 12, 16 en 24 threads. De Cassandra-cluster, met de \texttt{s.vcf.gz}-dataset, bestond uit 9 nodes, met replicatiefactor 3.

\newpage
\subsubsection{Resultaten}

\begin{table}[!h]
\begin{tabular}{@{}lllll@{}}
\cmidrule(l){2-5}
                                       & \multicolumn{2}{|l|}{\textbf{Query 1}}                                                      & \multicolumn{2}{l|}{\textbf{Query 8}}                             \\ 
\midrule
\multicolumn{1}{|l|}{\textbf{\# Threads}}  & \multicolumn{1}{l|}{\textbf{Subqueries [s]}}               & \multicolumn{1}{l|}{\textbf{Totaal [s]}} & \multicolumn{1}{l|}{\textbf{Subqueries [s]}} & \multicolumn{1}{l|}{\textbf{Totaal [s]}} \\ \midrule
\multicolumn{1}{|l|}{\textbf{4}} & \multicolumn{1}{l|}{$6.32 \pm 0.07$} & \multicolumn{1}{l|}{$1850.51 \pm 60.21$}             & \multicolumn{1}{l|}{$1298.77 \pm 8.52$}          & \multicolumn{1}{l|}{$1313.82 \pm 9.00$}                        \\
\multicolumn{1}{|l|}{\textbf{8}} & \multicolumn{1}{l|}{$6.22 \pm 0.05$} &\multicolumn{1}{l|}{$949.67 \pm 1.95$}            & \multicolumn{1}{l|}{$699.02 \pm 1.46$}          & \multicolumn{1}{l|}{$713.44 \pm 1.63$}                        \\
\multicolumn{1}{|l|}{\textbf{12}} & \multicolumn{1}{l|}{$6.21 \pm 0.09$} &\multicolumn{1}{l|}{$679.06 \pm 3.37$}            & \multicolumn{1}{l|}{$544.42 \pm 2.09$}           & \multicolumn{1}{l|}{$558.84 \pm 2.61$}                        \\
\multicolumn{1}{|l|}{\textbf{16}} & \multicolumn{1}{l|}{$6.25 \pm 0.07$} &\multicolumn{1}{l|}{$562.33 \pm 6.50$}                 & \multicolumn{1}{l|}{$440.64 \pm 0.83$}                    & \multicolumn{1}{l|}{$454.98 \pm 1.03$}                        \\
\multicolumn{1}{|l|}{\textbf{24}} & \multicolumn{1}{l|}{$6.34 \pm 0.13$} &\multicolumn{1}{l|}{$443.46 \pm 6.85$}          & \multicolumn{1}{l|}{$353.30 \pm 1.68$}           & \multicolumn{1}{l|}{$367.29 \pm 2.08$}                        \\
\bottomrule
\end{tabular}
\caption{Meetresultaten voor de executietijd van GEMINI-queries in een Cassandra-databank, voor een variabel aantal gebruikte threads.}
\end{table}

\subsubsection{Reflectie}

Uit de resultaten voor query 1 blijkt dat de runtime van eenvoudige queries met een resultaat met hoge kardinaliteit lineair afneemt met het aantal gebruikte threads. Gezien de evaluatie van de subqueries (in dit geval slechts 1) niet geparallelliseerd wordt, is er hierin weinig verschil te merken. Het verschil in de totale executietijd is dus volledig toe te schrijven aan verhoogde doorvoer bij het opvragen van de resultaten.\\
Bij query 8 is er een gelijkaardige duidelijke daling van de executietijd merkbaar. Het valt niet eenduidig te zeggen in welke mate dit toe te schrijven is aan een verhoogde leesdoorvoer, dan wel aan parallellisatie van de vele set-operaties die nodig zijn om de resultaten van de subqueries te parallelliseren. Vermoedelijk spelen beide een rol.

\subsection{Experiment 5: Schaalbaarheid \texttt{gemini query} met grootte Cassandra-cluster}
\label{exp5}

\subsubsection{Opzet}

In dit opzet willen we onderzoeken hoe de grootte van de gebruikte Cassandra-cluster de performantie van queries be\"invloedt. We verwachten een hogere leesthroughput en daardoor een positief effect op zowel het ophalen van het finale resultaat als bij het concurrent evalueren van meerdere subqueries.

\subsubsection{Omstandigheden}

Voor dit experiment hebben we query 1 en 8 uit experiment 3 (\ref{exp3}) hergebruikt: query 1 omwille van de grootte van het finale resultaat, query 8 omwille van de vele subqueries die in parallel ge\"evalueerd worden. We meetten de executietijd bij het doorzoeken van de \texttt{s.vcf.gz}-dataset in een Cassandra-cluster van 5, 7, 9 en 11 nodes. GEMINI benutte in dit experiment 24 threads.

\subsubsection{Resultaten}

\begin{table}[!h]
\begin{tabular}{@{}lllll@{}}
\cmidrule(l){2-5}
                                       & \multicolumn{2}{|l|}{\textbf{Query 1}}                                                      & \multicolumn{2}{l|}{\textbf{Query 8}}                             \\ 
\midrule
\multicolumn{1}{|l|}{\textbf{\# Nodes}}  & \multicolumn{1}{l|}{\textbf{Subqueries [s]}}               & \multicolumn{1}{l|}{\textbf{Totaal [s]}} & \multicolumn{1}{l|}{\textbf{Subqueries [s]}} & \multicolumn{1}{l|}{\textbf{Totaal [s]}} \\ \midrule
\multicolumn{1}{|l|}{\textbf{5}} & \multicolumn{1}{l|}{$ $} & \multicolumn{1}{l|}{$ $}             & \multicolumn{1}{l|}{$ $}          & \multicolumn{1}{l|}{$ $}                        \\
\multicolumn{1}{|l|}{\textbf{7}} & \multicolumn{1}{l|}{$6.19 \pm 0.09$} &\multicolumn{1}{l|}{$484.47 \pm 11.67$}            & \multicolumn{1}{l|}{$582.15 \pm 0.83$}          & \multicolumn{1}{l|}{$596.30 \pm 1.46$}                        \\
\multicolumn{1}{|l|}{\textbf{9}} & \multicolumn{1}{l|}{$6.34 \pm 0.13$} &\multicolumn{1}{l|}{$443.46 \pm 6.85$}            & \multicolumn{1}{l|}{$353.30 \pm 1.68$}           & \multicolumn{1}{l|}{$367.29 \pm 2.08$}                        \\
\multicolumn{1}{|l|}{\textbf{11}} & \multicolumn{1}{l|}{$6.43 \pm 0.07$} &\multicolumn{1}{l|}{$363.32 \pm 5.83$}           & \multicolumn{1}{l|}{$373.85 \pm 0.91$}           & \multicolumn{1}{l|}{$388.49 \pm 0.96$}                        \\
\bottomrule
\end{tabular}
\caption{Meetresultaten voor de executietijd van GEMINI-queries in een Cassandra-databank, voor variabele grootte van de Cassandra-cluster.}
\end{table}

\subsubsection{Reflectie}

Voor query 1 is er een duidelijke daling van de queryduur merkbaar. Gezien de grootte van het resultaat van query 1 heeft die alle baat bij een verhoogde leesthroughput: de duur van het evalueren van de query evolueert niet met de grootte van de cluster, de duur van het ophalen van de finale 339819 rijen wel.\\
Query 8 verloopt ook sneller bij toenemend aantal nodes in de cluster, maar het effect is niet zo drastisch als bij query 1. Van 9 naar 11 nodes is er zelfs een stijging merkbaar. De geringere stijging van de snelheid valt te verklaren door het lagere aantal (sub)queries dat gebeurt tijdens het evalueren van query 8 dan tijdens het ophalen van het resultaat van query 1. Query 8 voert 556 subqueries uit, terwijl query 1 per 50 rijen (van de 339819) een query zal uitvoeren. Dit bevestigt de hypothese dat de evaluatie van query 8 eerder CPU- dan I/O-bound is.

\subsection{Experiment 6: Impact dataduplicatie}
\label{exp6}

\subsubsection{Opzet}
Het doel van dit experiment is de impact van de dataduplicatie in het dataschema van ons ontwerp te bestuderen. Daarvoor hebben we de gebruikte schijfgrootte van de Cassandra-databanken voor enkele genoomdatasets vergeleken met de grootte van de overeenkomstige SQLite-databanken.

\subsubsection{Omstandigheden}

We hebben de grootte van de Cassandra-databases per node vergeleken voor de \texttt{s\_1092.vcf.gz}-,  \texttt{m\_1092.vcf.gz} en \texttt{l\_1092.vcf.gz}-datasets. We gebruikten een Cassandra-cluster van 12 nodes met replicatiefactor 3. In onderstaande resultaten is de replicatiefactor weggerekend.

\newpage
\subsubsection{Resultaten}

\begin{table}[h]
\centering
\begin{tabular}{@{}llll@{}}
\cmidrule(l){2-4}
                & \multicolumn{2}{|l|}{\textbf{Cassandra}}                                                      & \multicolumn{1}{l|}{\textbf{SQLite}}                             \\ 
\cmidrule(l){1-4}
\multicolumn{1}{|l|}{\textbf{Dataset}}  & \multicolumn{1}{l|}{\textbf{Per node [GB]}}               & \multicolumn{1}{l|}{\textbf{Totaal [GB]}} & \multicolumn{1}{l|}{\textbf{Totaal [GB]}} \\ \midrule
\multicolumn{1}{|l|}{\texttt{s\_1092.vcf.gz}} & \multicolumn{1}{l|}{$3.88 \pm 0.14$}             & \multicolumn{1}{l|}{$46.54$}          & \multicolumn{1}{l|}{$1.3$}                          \\
\multicolumn{1}{|l|}{\texttt{m\_1092.vcf.gz}} & \multicolumn{1}{l|}{$ $}             & \multicolumn{1}{l|}{$ $}          & \multicolumn{1}{l|}{$1.9$}                          \\
\multicolumn{1}{|l|}{\texttt{l\_1092.vcf.gz}} & \multicolumn{1}{l|}{$ $}             & \multicolumn{1}{l|}{$ $}          & \multicolumn{1}{l|}{$4.6$}                          \\
\bottomrule
\end{tabular}
\caption{Grootte in GB van de Cassandra- en SQLitedatabanken voor verschillende genoomdatasets. Replicatie is niet meegerekend voor de Cassandra-databases.}
\end{table}

\subsubsection{Reflectie}

De impact van de gegevensduplicatie in ons dataschema is duidelijk erg groot: vergeleken met de overeenkomstige SQLite-databanken, zijn de Cassandra-databanken vele malen groter, zelfs zonder replicatie in rekening te brengen. Wat ook meespeelt, is dat in SQLite alle genotype-gegevens in gecomprimeerd formaat bewaard worden, terwijl die in Cassandra gedecomprimeerd en bovendien meerdere keren (1 keer in de \texttt{variants}-tabel, 1 keer in de genotype-hulptabellen van de \texttt{variants}-tabel) opgeslagen worden.

\section{Evaluatie: conclusie}

In dit hoofdstuk hebben we ons ontwerp voor een schaalbare genoomanalysetool op basis van GEMINI en Apache Cassandra getest en experimenteel gevalideerd.\\
Met behulp van de unit-tests van de originele implementatie van GEMINI hebben we bewezen dat onze Cassandra-implementatie op een correcte wijze de belangrijkste features van GEMINI, namelijk het inladen van genoomdata en die met een verrijkte SQL-syntax doorzoeken, ondersteunt.\\
Daarnaast hebben we ook de performantie van onze implementatie gemeten door ze aan een reeks benchmarks te onderwerpen en de resultaten hiervan te vergelijken met de prestaties van de SQLite-versie van GEMINI. \\
Bij het inladen van genoomdata is de SQLite versie een grootteorde sneller dan onze implementatie, voornamelijk door de grote dataduplicatie in ons ontwerp. Dankzij de uitbreidbaarheid van ons ontwerp is dit echter een eenmalige kost.\\
Uit onze experimenten blijkt ook dat de duplicatie van gegevens in ons dataschema en het ongecomprimeerd opslaan van de genotype-informatie ervoor zorgen dat onze implementatie ettelijke malen meer schijfruimte vereist dan de oorspronkelijke SQLite-implementatie.\\
Het uitvoeren van eenvoudige SQL-queries verloopt ook sneller in de SQLite-implementatie, omdat deze niet over het netwerk moet communiceren om de resultaten op te halen.\\
Complexere queries met beperkingen op de genotypes van samples verlopen echter 1 tot 2 grootteordes sneller in onze implementatie: waar GEMINI in SQLite alle variants \'e\'en voor \'e\'en moet overlopen, kan GEMINI met Cassandra met enkele specifieke subqueries de kandidaat-variants opvragen en ze met set-operaties combineren tot het juiste resultaat.\\
We hebben ook de invloed van parallellisatie en de grootte van de gebruikte Cassandra-cluster op de performantie van onze implementatie onderzocht. Hieruit blijkt dat zowel het inladen als queryen van genoomdata baat heeft bij een groter Cassandra-cluster. Het queryen verloopt ook bijna lineair sneller naarmate meer threads worden gebruikt, zowel voor CPU- als I/O-intensieve queries.\\


