\chapter{Conclusie}

De kost om DNA te sequencen is de afgelopen 15 jaar drastisch gedaald, met een boom in de bioinformatica tot gevolg. Nu \emph{genomen} van steeds meer organismen almaar sneller ontleed kunnen worden, stelt dit hoge eisen aan de technologische infrastructuur om al de hieruitvolgende genoomdata op te slaan en effici\"ent te verwerken. De belangrijkste bekommernissen zijn schaalbaarheid naar enorme datasets en snelle, interactieve analyse van diezelfde datasets.\\
Om een gelijkaardige explosieve groei van datahoeveelheden op te vangen, hebben webbedrijven het afgelopen decennium de NoSQL- en NewSQL-datastores ge\"introduceerd: gegevensopslagsystemen die elk in verschillende aspecten afwijken van het traditionele relationele databankmodel om beter te schalen naar grote datahoevelheden in de diverse toepassingsgebieden van webbedrijven.\\
In dit eindwerk hebben we onderzocht hoe NoSQL- en NewSQL-technologie\"en van nut kunnen zijn in het genoomanalyseproces. We hebben eerst een vergelijkende studie van 6 verschillende NoSQL- en NewSQL-systemen uitgevoerd en voor elk van deze systemen ingeschat voor welk aspect van het genoomanalyseproces ze nuttig kunnen zijn. Sommige systemen, zoals Apache Cassandra, bleken geschikt voor de grootschalige opslag en analyse van reeds ontlede \emph{genomen}, andere, zoals Cloudera Impala, voor uiterst gedetailleerde en performante analyse van kleinere datasets, en nog andere, zoals VoltDB, eerder voor de opslag en performante verwerking van snel veranderende data in de sequencing-pijplijn zelf.\\
Deze kennis hebben we toegepast op de genoomanalysetool GEMINI: een softwareframework dat onderzoekers toelaat genoomdata van grote populaties proefpersonen in een SQLite-database in te laden, die met zeer uiteenlopende gegevens over het menselijk genoom te annoteren, en hier vervolgens uitgebreide queries in een verrijkte SQL-syntax op uit te voeren.\\
Om GEMINI beter te laten schalen naar grotere datasets hebben wij een ontwerp voorgesteld met een onderliggende Apache Cassandra- i.p.v. SQLite-databank, met een aangepast, incrementeel uitbreidbaar dataschema en een querymechanisme in de applicatielaag om de oorspronkelijke functionaliteit van GEMINI te bewaren.\\
We hebben dit ontwerp ook ge\"implementeerd, en experimenteel aangetoond dat het ten opzichte van de originele SQLite-versie van GEMINI veel gedupliceerde data bevat, daardoor ook tot 10x trager genoomdata inlaadt, maar voor queries op grote datasets vaak sneller is dan de originele versie. Voor complexe queries zoals die in onderzoek naar erfelijke ziektes bijvoorbeeld voorkomen, is onze implementatie zelfs consequent en tot 150x sneller dan de originele versie van GEMINI.\\
Uit dit eindwerk kunnen we besluiten dat er zeker een punt te maken valt voor NoSQL- en NewSQL-technologie\"en in de genoomanalyse. Ons concreet ontwerp voor GEMINI met Cassandra biedt ondanks enkele beperkingen perspectieven, en naarmate de bioinformatica er in de nabije toekomst verder op vooruitgaat, zal de vraag naar gelijkaardige oplossingen enkel toenemen. Het laatste woord over dit onderwerp is dan ook nog bijlange niet gesproken.