\chapter{Achtergrond genoomanalyse}
\label{dna_dummies}

{\color{red} TODO: definitie variant}
%TODO def variant? http://www.completegenomics.com/FAQs/Variant-Calls-SNPs-and-Small-Indels/

Omdat deze thesis binnen het vakgebied van de genoomanalyse ligt, volgt in dit hoofdstuk een beknopte inleiding in de wereld van DNA, genomen, genen en aanverwanten, met speciale aandacht voor het DNA van de mens.\\

Desoxyribonucle\"inezuur, kortweg \textbf{DNA}, is de chemische stof die als belangrijkste drager van erfelijke informatie dient in bijna alle levende wezens. DNA bestaat uit twee spiraalvormige, rond elkaar gewikkelde strengen, ook gekend als een dubbele helix. Deze twee strengen zijn elk lange ketens van zogenaamde \textbf{nucleotidebasen}. Zo zijn er precies 4: adenine (A), thymine (T), guanine (G) en cytosine (C). De twee strengen zijn aan elkaar gekoppeld door paren van deze basen. Er komen slechts 2 verschillende baseparen voor: AT en GC. Beide strengen zijn dus complementair \cite{genome_gov} \cite{nature_scitable}.\\

Het \textbf{genoom} duidt op de volledige set DNA van een organisme. Het menselijk genoom bestaat uit 3 miljard baseparen. Het grootste deel van DNA bevindt zich in lichaamscellen in de vorm van \textbf{chromosomen}. Menselijke cellen bijvoorbeeld tellen 23 chromosomen.\\
\textbf{Genen} liggen op deze chromosomen en bestaan uit \'e\'en of meerdere DNA-sequenties die de informatie encoderen voor de productie van \'e\'en of meerdere prote\"inen. Het deel van genen dat effectief deze informatie encodeert, heet het \textbf{exon}; alle exonen samen worden aangeduid met de term \textbf{exoom} en zijn goed voor ongeveer 1.5\% van het totale DNA \cite{broad_exome}.\\
Genen kunnen verschillende invullingen hebben: sommige genen hebben verschillende vormen die op dezelfde positie op het chromosoom liggen, \textbf{allelen} genaamd. Organismen zoals de mens hebben voor elk gen 2 allelen, \'e\'en overge\"erfd van elke ouder. Zijn deze twee allelen gelijk, dan is het organisme \textbf{homozygoot} voor het gen in kwestie, anders is het \textbf{heterozygoot}.

Het \textbf{genotype} is de specifieke chemische invulling van het DNA. Daartegenover staat het \textbf{phenotype}, dat duidt op de waarneembare fysieke eigenschappen van een levend wezen. Dit kunnen zeer uiteenlopende eigenschappen zijn, gaande van haar- en oogkleur tot het al dan niet lijden aan een bepaalde aandoening.\\

\textbf{DNA sequencing} is het bepalen van de exacte sequentie van nucleotidebasen in een streng DNA. De vandaag meest gebruikte sequencing methode genereert reads van 125 opeenvolgende nucleotiden, en miljarden reads tegelijkertijd. Om de segmenten horende in \'e\'en langer stuk DNA aan elkaar te kunnen linken, is het nodig vele overlappende segmenten te lezen, die vervolgens met elkaar gealigneerd worden. Elke nucleotide moet dus meerdere keren gelezen worden om een goede accuraatheid van het uiteindelijke resultaat te bekomen. De \textbf{depth} van een nucleotide, of bij uitbreiding een groter stuk DNA, is het aantal keren dat een nucleotide gelezen werd tijdens het sequencingproces. De \textbf{coverage} de gemiddelde depth over de hele DNA-streng die gesequenced werd en dus een maat voor de resolutie en accuraatheid van het sequencingproces. De uiteindelijke sequence-data worden vaak opgeslagen in SAM-files (of BAM, het binaire equivalent) en het is op basis hiervan dat \textbf{variant calling} gebeurt: het bepalen welk genotype een proefpersoon precies heeft voor een variant.\\\\
De resultaten van het variant calling-proces worden opgeslagen in \textbf{VCF}-bestanden (Variant Call Format). Het VCF-formaat bevat naast meta-informatie een lijn voor elke geobserveerde variant, met daarin optioneel informatie over het genotype van \'e\'en of meerdere proefpersonen.\\
In onderzoek naar het DNA van proefpersonen is ook nog andere informatie van tel dan enkel hun genotypes: gegevens als het geslacht of phenotype van proefpersonen kunnen uiterst relevant zijn voor onderzoek naar bijvoorbeeld genetische ziektes. In experimenten die het DNA van meerdere proefpersonen analyseren, is het ook bijzonder interessant de onderlinge verwantschappen tussen de proefpersonen te kennen. Samen met de genotype-informatie is het dan mogelijk erfelijkheidspatronen in de genoomdata te bestuderen. Dergelijke gedetailleerde informatie over de proefpersonen zit echter niet in de VCF-files, maar kan gespecifieerd worden in zogenaamde pedigree-files (\textbf{PED}-files). 